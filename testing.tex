%!TEX root = ./seminarpaper.tex

\chapter{Testing the Implementation}

	As defined in the task, the values of the statistical distributions implemented in \setlx\ should be tested. Testing should be done in two different ways - integration tests and regression tests. Both tests are described in the sections below.


\section{Integration Tests}

	The \setlx\ interpreter itself has two special test folders, one for regression tests (\lstinline{example_SetlX_code}) and one for integration tests (\lstinline{integrationTests}), containing code of all kinds of tests. Integration tests are tests written in the \setlx\ language, that ensure that implemented functions really exist and that they really deliver the correct solution. The folder for the integration tests is located in the \lstinline{interpreter} folder (listing \ref{lis:integrationTests}). 

	\begin{center}
		\begin{lstlisting}[caption={Path to Integration Tests}, label={lis:integrationTests}]
			setlX
			|-- example_SetlX_code
			`-- interpreter
				|-- integrationTests
				`-- stat_addon
		\end{lstlisting}
	\end{center}

	For the statistical distributions asset a new file \lstinline{stat.stlx} was created inside the \lstinline{integrationTests} folder. This file holds the tests for all 24 implemented distributions. At the beginning of the file, a special file named \lstinline{baseIntegrationTest.stlx} is loaded that will validate the tests. For each distribution, a \setlx\ procedure was written. The following listing shows this with an example. 

	\begin{center}
		\begin{lstlisting}[caption=Test File Example, language=setlx, label=lis:exampleCode]
		statChiSquared := procedure() {
			if (stat_chiSquared(2,3) == 0.2075537487102973) {
				correct := true;
			} else {
				print("Error is in: stat_chiSquared(2, 3) == $0.2075537487102973$");
				correct := false;
			}

			validateForTestCase("statChiSquared")
				.that(
					correct
				).is(
					true
				);
		};
		statChiSquared();
		\end{lstlisting}
	\end{center}

	After the procedure is called, the implemented distribution will be called with some parameters. An \lstinline{if} statement checks whether the result of the function call matches the number in the statement. Depending on the result, the test will pass or fail.


\section{Regression Tests}

	Regression tests basically check that everything still works the same. This is especially very important after changes have been done.

	To test all regression tests within the folder, a script named \lstinline{test_all_examples} needs to be executed. The script will first run all integration tests and after validating these tests, the script will check the regression tests, by searching for files that end in \lstinline{.stlx} and files that have the same basename, but end in \lstinline{.stlx.reference}. Once both files are found, the \lstinline{.stlx} file will be executed and the result will be compared with the corresponding \lstinline{.stlx.reference} file, that definitely contains the solution of the \setlx\ program. For the comparison \lstinline{diff} is used. If the result matches the solution, the diff needs to be empty. If so, the next source file will be executed. If not, the script will stop and the result needs to be analyzed.

	The idea was to expand this scheme for the tests of the implemented statistical distributions. Since each asset has its own test folder, it was decided to create a new folder named \lstinline{stat_test_code}, which would then contain all test and reference files. The resulting folder structure of \lstinline{example_SetlX_code} looks like the following:
	
	\begin{center}
		\begin{lstlisting}[caption={Folder Structure \lstinline{example\_SetlX\_code}}, label={lis:regressionTests}]
			example_SetlX_code
			|-- animation_testcode
			|-- converted_Setl2_code
			|-- performance_test_code
			|   `-- results
			|-- plotting_test_code
			|-- simple_examples
			|   `-- gfx_addon
			`-- stat_test_code
		\end{lstlisting}
	\end{center}

	In total 24 statistical distributions were implemented. For each function, a test and a reference file were created, so that there are 48 files in total. All test files follow the same pattern, which is shown in the following listing. The reference files only contain the solution of the program.

	\begin{center}
		\begin{lstlisting}[caption=Test File Example, language=setlx, label=lis:exampleCode]
		// Example for stat_normal(x, mu, sigma)

		statNormal := procedure(x, mu, sigma) {

			print(stat_normal(x, mu, sigma));
		};

		x     := 2;
		mu	  := 3;
		sigma := 9;

		statNormal(x, mu, sigma);
		\end{lstlisting}
	\end{center}

